%%%%%%%% 結び %%%%%%%%%%%%%%%

%\chapter{まとめ}
%本稿では,マッチング処理制約のある4工程FFSにおいて不良部品の発生が起こるという条件の元で,マッチング処理制約と不良部品の検査工程数によるFFSの性能に対する影響を調査,考察した.
%その結果マッチング処理制約による影響はFFSのバッファの混雑度によって変化すること,また不良部品の検査・廃棄タイミングとしては”第1工程直後と第4工程直後”で検査行うとよりスループットの低下が抑えられるという結果を得ることができた.
%
%本研究で扱ったFFSモデルは,バッファサイズが小さいため,ブロッキング確率が比較的高くスループットが低い状況を想定している.
%今後の課題としてはバッファサイズをより大きなものに変更し,バッファサイズがスループットに与える影響しているのかを調査することがあげられる.
%また,FFS以外の様々な生産システムに対し不良部品が与える影響を調査する予定である.

\section{Summary}
In this paper, we investigated and examined the influence on the performance of FFS by the match processing constraint and the number of inspection step of defective parts under the condition that defective parts occur in 4 step FFS.
As a result, the influence of the match processing constraint varies depending on the congestion degree of buffer of the FFS, and as the inspection removal timing of defective parts is carried out "immediately after the first and fourth step," the lowering of the throughput is suppressed.

The FFS model we used in this research assumes a situation where the blocking probability is relatively high and the throughput is low because of small buffer size.
For the future work, we will change the buffer size to a larger one and investigate whether the buffer size affects the throughput.
We will also investigate the impact of defective parts on various production systems other than FFS.
